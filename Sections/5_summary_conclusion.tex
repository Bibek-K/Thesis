\newpage 
\section{SUMMARY AND CONCLUSION}
\subsection{Introduction}
This section presents a comprehensive summary of the major findings of this research, draws conclusions, and offers recommendations based on the analysis of the study. As outlined in the earlier sections, the primary objective of this study is to examine how electoral characteristics influence the fiscal and economic decision-making processes of local governments in Nepal. To achieve this objective, data was collected on the electoral characteristics of local levels based on the 2017 local level elections. Additionally, fiscal and economic data was gathered for all 753 local governments in Nepal spanning the years 2017 to 2021. However, due to the lack of standardization in the 2017 and 2018 data provided by the government, these years' data were excluded from the analysis. Instead, the study utilized data from the years 2018 to 2021, leading up to the subsequent local level election, for a more accurate analysis.
\subsection{Summary and Conclusion}
The role of elections in democratic systems is well-documented in economic and political science literature, particularly regarding their influence on the behavior of elected officials. Elections serve as a crucial mechanism for ensuring fiscal and economic discipline among elected officials through processes of accountability, which include both sanctions and selection. This study has found that the local governments in Nepal exhibit fiscal accountability as a result of their electoral characteristics. According to the regression analysis presented in the earlier sections, the Herfindahl-Hirschman Index (HHI) of electoral competition at the mayoral level significantly impacts various aspects of fiscal and economic discipline.\\
Specifically, as the level of electoral competition increases, the HHI index decreases. This relationship was anticipated to result in a negative correlation with revenue variables and a positive correlation with expenditure variables. The regression results confirmed these predictions, establishing that local governments in Nepal are indeed influenced by the level of electoral competition. Moreover, the findings of this study highlight the importance of education in relation to fiscal and economic discipline. Higher levels of education within a local government are associated with positive and significant effects on revenue variables and negative and significant effects on expenditure variables. In other words, as the educational level of the local population increases, the fiscal and economic indicators also improve.\\
Our finding also reveals that having divided control of the local government do not significantly effect the income variables, and only weakly effect the expenditure variables. It suggests that power sharing is not inherently good or bad in context of Nepals federalism. Furthermore, the difference of fiscal and economic performance between rural municipalities and municipalities is significant, as our finding shows that rural municipalities significantly collect less internal revenue and spends more on current expenditures compared to municipalities. 


We can conclude that .......Based on the findings of this study, several recommendations can be made to enhance the fiscal and economic performance of local governments in Nepal. These recommendations are aimed at improving electoral processes, leveraging educational advancements, and addressing the disparities between different types of municipalities.
\begin{enumerate}[label=\roman*.]
    \item  \textbf{Strengthen electoral competition:}This study highlights the significant impact of electoral competition on fiscal and economic discipline. To foster a more competitive electoral environment, it is recommended that reforms be introduced to enhance transparency and fairness in elections. This could involve improving electoral processes, ensuring that candidates have equal opportunities, and implementing measures to prevent electoral fraud. Increasing the level of competition can help ensure that elected officials remain accountable to their constituents and are motivated to manage fiscal resources more effectively.
     \item \textbf{Promote educational attainment:} The positive correlation between higher education levels and improved fiscal and economic indicators suggests that investing in education can yield substantial benefits for local governments. Policies aimed at enhancing educational opportunities and increasing the educational attainment of the local population should be prioritized. This could include expanding access to quality education, providing vocational training programs, and supporting adult education initiatives. By improving the overall educational level, local governments can benefit from a more informed electorate and better governance outcomes.
   \item \textbf{Further research on power sharing:} This study found that divided control of local governments does not significantly affect income variables and only weakly affects expenditure variables. This suggests that power-sharing arrangements may not have a clear-cut impact on fiscal performance. It is recommended that further research be conducted to explore the nuances of power-sharing arrangements and their effects on governance. In the meantime, policymakers should consider the specific context of each local government when designing power-sharing structures, ensuring that they are tailored to the needs and circumstances of the local population. 
  \item \textbf{Address disparities between rural and urban municipalities:} The significant difference in fiscal and economic performance between rural municipalities and urban municipalities calls for targeted interventions. Rural municipalities, which tend to collect less internal revenue and spend more on current expenditures, may benefit from targeted financial support and capacity-building initiatives. The government should consider implementing policies that address the unique challenges faced by rural municipalities, such as providing financial incentives, improving infrastructure, and supporting local economic development initiatives. Additionally, efforts should be made to improve revenue collection mechanisms in rural areas to reduce disparities and promote more equitable development. 
 \item \textbf{Enhance fiscal management practices:} To improve overall fiscal discipline, local governments should be encouraged to adopt best practices in fiscal management. This includes implementing robust budgeting processes, ensuring transparency in financial reporting, and regularly monitoring and evaluating fiscal performance. Training and capacity-building programs for local government officials can further enhance their ability to manage fiscal resources effectively.   \end{enumerate}
In conclusion, implementing these recommendations can help local governments in Nepal improve their fiscal and economic performance, enhance accountability, and promote more equitable development across different regions. By addressing the challenges identified in this study and leveraging the opportunities for improvement, policymakers can contribute to the overall betterment of governance and development outcomes in Nepal. 

Write a brief paragraph on Possible extension, concept and method both highlight but subtle ways. 
