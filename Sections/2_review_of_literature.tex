\newpage
\section{REVIEW OF LITERATURES} 
\subsection{Introduction} \vspace{-3mm}
This chapter examines several literatures related to the theme of elections, local governance, and their relationship as it pertains to fiscal and economic disciplines. As federalism is relatively a new practice for Nepal, and its' formal initiation was less than a decade ago, this field contains paucity of empirical research in the national context. However, there are plethora of literature on federalism both from the theoretical perspective as well as in the empirical and international context. Thus, this study first looks at the theoretical overview of differences in the local level fiscal and economic disciplines in the federalist system, followed by the empirical review of such differences. This section concentrates the international literature within the empirical review and finally presents the literature gap that this study further focuses upon. \vspace{-5mm}
\subsection{Theoretical Review} \vspace{-3mm}
The literatures in federalism and fiscal decentralization can be grouped into first generation model and the second generation theory. The FGT approach to the role of federalism and inequality among the federal units are characterized by the writings of chiefly, Musgrave, Oates, Tiebout, and Buchanan. The second generation models, characterized by using theories from microeconomics in federalism analysis, are chiefly influenced by the writings of Weingast, Seabright, Lockwood, and  Coate. On balance, FGT theorists investigating fiscal federalism tended to focus on tax assignment among the subnational units and associate the process of fiscal decentralization with an enhancement in the overall degree of public sector responsiveness to a public demand and, ultimately, to an improvement in the economic efficiency of public economic activities by better linking resource allocation with public preferences. Whilst the SGT of fiscal federalism focuses on how political and institutional arrangements within subnational units interact and affects the economic and fiscal outcomes of the nation. \par
Tied with both FGT and SGT of fiscal federalism is the electoral accountability aspect. \citeA{Downs1957} is one of the early and  seminal works examining how the electoral decisions are made and how the elected officials are held accountable. Inspired by the earlier work of Schumpeter, Downs developed the rational model which influences both, the voting turnout of the electorate, and the actions of incumbent and opposing candidates. Focusing on the available knowledge available, i.e., perfect or imperfect, and the cost of information, Downs' model posit that while voters decide their votes based on the rational utility assumptions, political parties and candidates formulate policies strictly as a means of gaining electoral victory. Thus, the elected officials are accountable to the electorate in increasing their utility through policy formulation and carrying out procedures, and apart from this social function, the incumbent also has the private motive, namely, to attain and maintain the income, power, and prestige of being in office. Thus, elections do affect the policy choice and  formulation by the winners as they are accountable to increase the perceived general welfare level of the electorate. \par
\citeA{Baron1994} presents a model of electoral competition where electorate and candidates behaviors, influenced by ideologies, interest groups, campaign contributions, and information could manifest in an electoral equilibria. Investigating the interaction between the populace and candidates through various mediums, this model distinguishes the electorate as either informed or uninformed about the policy positions and ideologies presented by the political parties and the candidates. In this model, voters provides not just the votes in the election, but also makes up the interest groups, who, according to their ideological similarities,  provide campaign contributions to the candidates and the political parties. The accountability of the elected officials are categorized through particularistic policies, which benefits only the selected group of the electorate, or collective policies, which benefits all electorate. Thus, in this model, the elected officials are accountable and it is within their realm of choices to fulfill their accountabilities, either to particularistic or collective electorate.\par
\citeA{Grossman1996} investigates the dynamics between the competing parties as well as between candidates and electorate. This model examines the determinants of the policy positions, campaign contributions, vote counts, and policy compromise within an accountability framework. Ignoring the prospect of coalition formation, and limiting the analysis to a two-party system, this model assumes two types of voters, knowledgable voters who vote based on the ideologies of the candidates and are more prone to demand for accountability, and impressionable voters, with no set ideological preferences, who are susceptible to campaign advertisements and other forms of electioneering. Political equilibrium is achieved, either through influence motive or electoral motive or both, where candidate set their policy platforms maximizing their votes, and perspective voters set to be the recipient of maximal patronage that could be delivered to the party loyalists. Using a game theoretic approach to platform setting and voting decision, this paper develops six propositions, some of which achieves Nash equilibrium.\par
\citeA{Alesina1996} tackle the crucial topic of fiscal consolidation and long-term fiscal balance in economic policy, which is a matter of great concern for emerging countries, transition economies, and members of the OECD. The two primary focuses are deficit reduction (particularly in high-debt-to-GDP nations) and the necessity of significant welfare and social security system reforms for long-term fiscal stability. The benefits of inflation targeting and central bank independence have been widely discussed in monetary policy discussions; however, there has been less theoretical and empirical research done on budget procedures and institutions.This study analyzes possible institutional adjustments and examines whether budget processes have an impact on budget balances and their composition. In order to centralize and streamline the budget process, it finds that budget processes do important and suggests strengthening the executive's role—specifically, that of the treasury minister—and improving transparency.Budget preparation, approval, and execution are governed by rules and regulations, as revealed by theoretical insights into budget institutions. This study makes a distinction between procedural requirements (like voting rules), numerical objectives (like balanced-budget regulations), and transparency rules. Although numerical objectives have the ability to promote fiscal discipline, they may also cause rigidity and less transparency, which might compromise their efficacy. There is a trade-off between hierarchical and collegial methods in the procedural norms that govern budget creation and approval.Collegial processes encourage checks and balances but may postpone budgetary adjustments, whereas hierarchical methods may improve fiscal discipline but may also benefit the majority. The influence of budget openness is also covered in the study, with an emphasis on how a lack of transparency can impede fiscal discipline by fostering innovative accounting procedures and raising doubts about the health of public finances.Research on nations in Europe, Latin America, and the United nations has shown empirical evidence that more fiscal restraint is linked to budgetary practices that are transparent and hierarchical. The hypothesis that more transparent and hierarchical processes promote financial discipline is supported by studies by previous literatures. Measuring transparency and comprehending how different budgetary procedures affect fiscal performance are still difficult tasks, though. Different budget rules and processes may have an impact on how quickly and efficiently governments respond to fiscal shocks. This paper concludes that efficient budgetary practices should avoid getting in the way of fiscally responsible governance, even if no organization can completely avert deficits once they are identified.\par
\citeA{Smart2013} posit that periodic elections are the main medium through which the voters can hold elected officials and candidates accountable and the term limit imposition breaks such accountability by restricting voters' ability to reward politicians with reelection votes. Simultaneously, this study argues that term limits, which negates the accountability and disciplinary control of voters over the elected officials, can serve the interest of voters by increasing their welfare. This study formulates a model in which elected officials and opposing candidates have private information about the effectiveness of policies, which can be of two types, namely, public-spirited, in which payoffs coincides with those of the populace, and others, which have biased preferences. This study develops a political agency model where incumbent makes a binary policy decision on behalf of a representative voter. The model developed is a game between an infinitely loved representative voter and a sequence of elected politicians in an infinite period. This study compares single, two, and more term limit regulations effect, analyzes the implications of an endogenous pool of challengers, and examines asymmetric information between voters and incumbent while also analyzing the rents received and other utilities derived from implementing preferred policies by elected officials, and finally compares how the model developed compares with the existing empirical evidence. The study develops eight proposition, and argues that while accountability aspect is broken in case of term-limits, term-limits can decrease the rent-seeking and opportunistic behaviors of the elected officials.\par
\citeA{Foucault2017} explore the intricate relationship between political results, economic conditions, and tax laws. The authors focus on two main issues: whether voters hold incumbents responsible for bad economic performance regardless of the economic policies they implement, and if economic downturns make incumbents' election defeat inevitable. The authors offer a novel theoretical framework to investigate these issues, taking into account the combined effects of taxation policies, the definition of political authority, government ideology, and the state of the economy on election results. This theory suggests that taxation policies have an independent impact on election results, challenging the common notion that economic performance alone determines electoral success or failure.The study also makes use of a recently assembled dataset that contains data on economic indicators, elections, and government policies from both before and after the most recent financial crisis. The results show that a number of factors influence how taxation policies affect incumbent electoral success. In particular, these impacts differ according on the degree to which voters can clearly connect economic outcomes to the government in office, the government's ideological stance, and whether or not there was a recession in the year before an election.The study essentially emphasizes the complex link between voter behavior and economic management. It implies that people react not just to economic results but also to how fair and successful they believe the government's tax policy to be. Furthermore, the political environment, which includes the government's capacity to convey accountability for economic results and its ideological position on economic matters, moderates these impacts. This study adds to the body of knowledge on electoral accountability by providing a more in-depth analysis of the ways in which policy and economic issues interact to influence voter behavior. It emphasizes how crucial it is to take into account the larger political and economic backdrop when evaluating election results rather than depending just on economic metrics like GDP growth or unemployment rates.\par
The trade-offs between externalities and accountability, through governance and electoral uncertainty is examined by  \citeA{Aidt2017}. The main emphasis is on the impact of fiscal decision-making centralization on political responsibility, especially when externalities—both positive and negative—are present. The study concludes that greater rent extraction by federal politicians results from centralization's weakening of political accountability, particularly when negative externalities are present. In situations when negative externalities are substantial, the article also implies that decentralization could be preferable to centralization.The research analyzes these dynamics using a standard agency model with uncertainty in the governance. In this model, elections are used by diverse voter groups (the principals) to hold a politician (the agent) responsible. The model presents the idea of governance uncertainty, which is the situation in which a politician is unsure of which voter group will play a key role in determining the election's outcome because of things like random variations in voter turnout. As demonstrated by actual cases such as how rain affects voter participation in different elections, this uncertainty can have a major impact on election results. In terms of how much rent a federal politician may take, the research shows an imbalance between positive and negative externalities. Despite regionalism's incapacity to absorb externalities, federal politicians may be able to collect more rents in the event of negative externalities than their regional counterparts. This might potentially render federalism less advantageous than regionalism. Federalism, on the other hand, is often more advantageous when there are positive externalities as it internalizes them without permitting further rent-seeking by federal politicians. By clearly quantifying the trade-offs between electoral accountability and internalization of externalities and by emphasizing how governance uncertainty affects these trade-offs, it provides great insights into how accountability works in the federal settings. The implications for fiscal integration and disintegration are also covered in the study, and it is suggested that in some situations, the existence of negative externalities may make decentralization more advantageous.The study comes to the conclusion that while choosing between centralization and decentralization, one should take into account the kind of externalities and how political accountability is affected by uncertainty in governance. Specifically, it implies that decentralization could be more Pareto optimal when negative externalities are substantial, as centralization might encourage federal politicians to take undue rent.\par
Examining the impact of decentralization and devolution on democratic governance and electoral responsibility, \citeA{Charbonneau2021} contend that by bringing government closer to the people, decentralization might improve electoral accountability by strengthening voters' capacity to assign blame and hold local authorities accountable for their deeds. Their study also discuss some drawbacks, such as the adverse externalities that might result from decentralized systems without efficient supervision and coordination methods. They point out that decentralization can boost voter engagement and political participation by increasing the relevance and direct influence of local elections on the lives of individuals. Voters will be better equipped to make educated judgments on the performance of their representatives as a result of this close closeness to information. Their study also affirms voters becoming confused about which level of government is in charge of certain policies due to the complexity brought about by decentralization, which might weaken accountability. It talks about the double effects of decentralization: on the one hand, it gives local governments the flexibility to customize policies to meet the unique requirements of their communities, which promotes creativity and adaptability. However, it can also result in issues like unequal service delivery and capability gaps across local governments, which could compromise the efficacy of local governance.This work is consistent with previous empirical observations of the advantages and difficulties of decentralization, especially with regard to the ways in which local autonomy may improve democratic accountability and the standard of governance. It also supports the ideas of fiscal federalism covered by Oates (1999) in which the budgetary implications of decentralization play a critical role in determining its overall efficacy. In order to obtain the best governance outcomes, both texts stress the significance of striking a balance between autonomy and cooperation. In light of the current international discussions over the advantages of decentralization as a governance approach, the ideas from this chapter are very pertinent. Policymakers seeking to execute successful decentralization programs must grasp the trade-offs and dynamics outlined by by this study as nations and regions struggle with calls for increased local autonomy and better governance.To sum up, the study significantly enriches the decentralization literature by offering a nuanced perspective on how the devolution of power to local levels shapes local governance and electoral accountability. It emphasizes how crucial it is to create decentralized systems that support accountability, openness, and efficient service delivery—all of which are necessary for strong democratic institutions to function.\\
\vspace{-9mm} 
\subsection{Empirical Review} \vspace{-3mm}
There are about 27 countries with federal structure, albeit many of them do not have enough power decentralization or devolution from the central to the federal units. However, there have been multitudes of empirical studies on how the federal system of governance are accountable to the populace and to what extent does the behavior of electorate matters. Thus, this study presents the empirical review conducted in the international context analyzing federalism and the differences among the federal units. \par
\citeA{Alt1994} examines previous literatures on the fiscal effects of split administration in different countries, and concentrates on the ways in which institutional restraints and political arrangements affect budgetary results, especially in the wake of economic shocks. The focus of the study is to comprehend the implications of institutional regulations like balanced budget requirements and deficit carryover legislation, as well as the influence of political power, whether it be united or split, on fiscal policy. To capture the complexity of fiscal decision-making, the study estimate simultaneous equations for income and expenditures using pooled time-series cross-sectional data from 48 states of the US between 1968 and 1987. Partisan control, institutional restraints, and economic shocks are just a few of the carefully chosen research variables that are used to examine how these elements combine to affect budgetary results. The results show that during economic downturns, divided governments—especially those with split legislatures—find it difficult to restore fiscal balance, which results in protracted deficits. States with strict deficit regulations, on the other hand, have united governments that are better equipped to handle sudden increases in revenue while preserving a steady fiscal balance. The study also reveals subtle party distinctions, demonstrating how Republicans and Democrats react to variations in revenue, which are impacted by economic cycles and government assistance, in different ways. The findings, which are based on extensive empirical testing and simulations, show how institutional design and political control combine to significantly influence state budgetary results.Overall, by highlighting the significance of institutional and political elements in managing state finances and providing insights into the dynamics of state budgeting procedures, the research adds to the body of knowledge on dynamics between fiscal policy and electoral accountability.\par
\citeA{Besley1995} look at the economic impacts of governor term limits, with a specific emphasis on how they influence fiscal policies like taxes and state expenditure. This study's major goal is to determine if term limitations cause a fiscal cycle that results in large fluctuations in taxes and spending, particularly during periods when governors are ineligible for reelection (lame-duck periods) and thus do not have to be strictly accountable to their electorate. Using a data set from 1950 to 1986, the authors use panel data regression techniques to examine the association between governor term restrictions and budgetary results. Key variables in the analysis include state tax revenues, state expenditures, electoral cycle indicators (such as election proximity and the governor's ability to run for reelection), and disaster relief data from the Small Business Administration (SBA) as an exogenous factor that could influence fiscal decisions. The technique entails assessing the impact of term limits using indicator variables that correspond to different times in the electoral cycle, allowing the analysis to capture the subtle impacts of these cycles on budgetary behavior. For example, the study investigates whether taxes and expenditures vary between election years, election years preceding elections, and years when governors are lame ducks. Furthermore, the study examines the regional variance in the impact of term limits, distinguishing between Southern states and others, and further divides the analysis by party affiliation, with an emphasis on Democratic governors. The data indicate a separate fiscal cycle caused by term limitations. Governors who are ineligible for reelection tend to raise taxes and spend considerably in their final term, which is not found among governors who can still seek for office. This shows that lame-duck governors, free of the need to woo voters for future elections, are more inclined to pursue financially expanding policies. Furthermore, the study reveals that natural catastrophes, which necessitate an urgent and major governmental reaction, have a significant impact on budgetary behavior. Governors running for reelection tend to raise taxes and spend more aggressively in reaction to catastrophes, presumably to match voter expectations for disaster recovery, but lame-duck governors show no additional budgetary response beyond their usual practice. The paper also expands its analysis to include the broader economic ramifications of these fiscal cycles, with a particular emphasis on state income per capita. The findings show that states run by lame-duck Democratic governors had a considerable drop in per capita income, implying that the fiscal inefficiencies caused by term limits might have negative economic effects. The study closes by analyzing the possible inefficiencies induced by these fiscal cycles, namely the distortion in resource mobilization and public benefit provision that occurs when taxes and expenditures are drastically changed owing to political cycles rather than economic necessities. The research presents strong empirical evidence that governor term restrictions can result in inefficient budgetary measures, especially during lame-duck years. These findings highlight critical considerations regarding the trade-offs associated with term limits, which may minimize political entrenchment while creating budgetary distortions that might harm state economies.\par
 \citeA{Kalseth1998} tackle the crucial topic of fiscal expenditure and long-term fiscal discpline, which is a matter of great concern for developing and developed nations alike.The study uses two main methodologies: (1) Data Envelopment Analysis (DEA) to establish a benchmark for the minimum required administrative spending, which serves as a reference point to measure overspending, and (2) a bargaining model integrated with political structures to understand the dynamics of administrative services. The study used a Tobit regression model to examine the influence of political variables on the discrepancy between the minimum needed spending and actual spending.The kind of political alliance (minority, majority, or one-party majority, for example), socialist presence in municipal councils, and additional economic and demographic characteristics are important determinants. The results show that while socialist-oriented councils often have higher spending because they prioritize administrative services, split political control frequently boosts bureaucratic influence and results in higher administrative expense. The study emphasizes how difficult it is for voters to limit administrative expenditures through elections alone because coalition dynamics and party agendas have a big impact on spending trends. The study's final conclusion is that political structures are a major factor in influencing administrative spending. The study focuses on the excessive administrative spendings of the Norwegian local governments and examines the impact of different ideologies of ruling government and the coalition form of the government. Their findings suggest that when there is a coalition type government, the bureaucrats influences on fiscal and economic decisions increases and accountability of elected officials towards the electorate suffers.\par
\citeA{Besley2002} look at how political organizations and the media affect how responsive and accountable the state government of India is, especially when it comes to handling emergencies like natural disasters and food shortages. The main goal of this study is to assess how political and media developments affect government action and compare them to the contribution of economic development. While economic development is frequently regarded as essential, the study postulates that political and media elements may be more important in improving government reaction to crises.Panel data regression analysis is used in the technique, and year- and state-specific factors are controlled for using fixed effects models. The model investigates the link between several explanatory variables, such as economic, media, and political factors, and government actions, such as public food distribution and catastrophe relief spending. In order to evaluate the influence of shock factors on government actions, the analysis incorporates interaction terms between media or political variables and shock variables (such as shifts in food production and flood damage).The data covers the years 1958 to 1992 and includes information on several Indian states. Economic parameters include population density, urbanization, state income per capita, and federal government revenue. Political factors include election timing, political competitiveness, and voter turnout, while media variables include total newspaper circulation and its breakdown by language. To mitigate potential endogeniety issues, media ownership is used as an instrument for newspaper circulation through the use of instrumental variable approaches.The results show a positive correlation between increased newspaper readership and improved government reaction to flood damage and drops in food output. Since regional language newspapers are better at reaching disadvantaged communities and addressing local concerns, this effect is especially pronounced for them. Political variables also have a big impact: more political competition and more voter turnout are linked to more government engagement, especially when it comes to public food distribution. Comparatively speaking, media and political variables have a greater influence on government responsiveness than economic growth factors like state income and federal revenue.The findings indicate that political institutions and media development are critical for enhancing government responsiveness, and they may even have a greater impact than economic development. Because of its local emphasis, regional media and strong political institutions improve government response and accountability. The study emphasizes the significance of a strong political environment and a well-developed media in guaranteeing efficient administration and proposes that future research should examine the ways in which media development affects local government accountability and other policy domains in emerging nations.\par
\citeA{Martinussen2004} investigates the factors that determine electoral performance of candidates and parties in local Norwegian elections, focusing on both political and economic drivers. The prime objective of this study is to examine how different political characteristics and economic situations influence election support for incumbent parties. To do this, this study uses OLS regression and stepwise regression methods. The political variables examined include the form of government (one-party or coalition), the government's minority status, and the amount of support from national parties. Unemployment rates, changes in municipal fees and levies, and the scope and quality of local services are among the economic considerations made.The study concludes that political qualities have a significant influence on election performance. Specifically, one-party regimes suffer higher election losses than coalition administrations. This is because one-party administrations provide more transparent accountability, making it simpler for citizens to express unhappiness. Minority administrations, on the other hand, gain from electoral support because they may transfer blame for policy errors, making them less open to criticism. National party support is also important, as more national support translates into increased support for local chapters. Furthermore, the ideological alignment of the mayor and deputy mayor is important, with governments made up of ideologically similar parties being more likely to face negative voter assessment.In terms of economic drivers, the study finds that rises in unemployment and municipal taxes or levies reduce support for the incumbent mayor's party. For example, a one-percentage-point rise in unemployment results in a 1.12-point drop in incumbent support. Conversely, high levels of service coverage are positively related with electoral support, implying that voters favor improved access to services over cost savings.The study also presents a Change and Level Version Model, which investigates whether voters evaluate both changes in economic indicators and their relative levels when rating incumbents. This method emphasizes how voters compare local incumbents to other towns or national averages, which influences their decision-making. The study concludes that, while economic changes have a direct influence on election performance, the level of economic indicators in comparison to others is equally important.\par
\citeA{Cutler2008} examines the role of voters accountability demand and the complexities they face when trying to attribute responsibility for policy outcomes across multiple levels of government, in context of Canadian Federalism. The study underscores the challenges posed by Canada's decentralized federal structure, where overlapping responsibilities can lead to confusion and hinder voters' ability to discern which government is accountable for specific outcomes. This study criticizes the extant literature for their simplistic measurement of voters responsibilities and the neglect of the certainty with which voters make those responsibilities attributions. Using the data from the original survey research conducted during the provincial elections in Ontario and Saskatchewan in 2003, followed by a federal election in 2004, the study indicates that voters struggle to differentiate the roles of different levels of government and further indicates that attentiveness to politics only slightly improves the accuracy of attributions.This study, thus, highlights difficulties in holding incumbents accountable in a federal system of governance where policy formulation and execution takes place in a complicated manner. \par
\citeA{Eckardt2008}  examines how political responsibility affects local governments' performance in the context of Indonesia, with a particular emphasis on the ways in which different local government circumstances influence the provision of public services. The study's main goal is to investigate how political structures and procedures affect local governments' capacity to better serve the needs of the community and provide public services. The accountability framework, on which the research is based, asserts that efficient political accountability frameworks increase local governments' responsiveness and boost service quality.The study uses a variety of approaches, including OLS regression analysis, to investigates the accountability aspects of the local governments. Based on a sample of 177 districts, this study gathers information on a range of political, budgetary, and socioeconomic factors. The study looks at the relationship between perceived improvements in public services and political accountability indices such political fragmentation, corruption in accountability reports, community involvement, and information access.Every variable is measured in depth in the paper, the likelihood that two randomly chosen council members would belong to separate parties is used to measure political fragmentation variable, with larger values denoting more fragmentation.The percentage of households reporting instances of corruption in local executives' yearly reports is taken as in Corruption in Accountability Reports. Higher values indicate more community participation. Access to information is determined by the proportion of residents attending health planning meetings and following local elections. Per capita expenditures, the percentage of wage bills in local budgets, and the share of development expenditures on health and education are used to evaluate the state of the fiscal system. In addition, to measure the effect of fiscal transfers on service performance, the proportion of the general allocation grant in total revenues (SHDAU) is examined.The study's conclusions show that there are a number of important connections between service delivery and political accountability metrics. Perceived advances in public services are positively correlated with more community engagement and less political corruption; conversely, higher political fragmentation is correlated with poorer service satisfaction. The study also reveals that while the influence of fiscal variables like wage bills and income sources exhibits complicated and even conflicting outcomes, larger per capita expenditures typically increase service quality. The article concludes that the efficacy of political accountability systems has a notable impact on the performance of local governments. Better services are typically provided by governments that are less corrupt, less divided, and more receptive to citizen involvement. The study does point out that while though these connections offer insightful information, causality cannot be conclusively proven, and the outcomes might be impacted by other unobserved variables. In order to validate these results and investigate the long-term impacts of decentralization on local government performance, the report highlights the necessity for more research.\par
 \citeA{Gevasoni2010} explores the concept of rentier states at a subnational level of Argentina, proposing that regions heavily reliant on external rents, such as natural resources or federal transfers, tend to develop unique political and governance characteristics. The theory presented suggests that these regions are less accountable and more autocratic due to their economic structure, which minimizes the need for taxation and thus, reduces citizen engagement, subsequently decreasing the accountability of the elected officials. This study extends the concept of rentier to the subnational level, arguing that similar dynamics can be observed within regions or states. The background sets the stage for understanding how economic dependencies, especially those based on external rents, influence political institutions and governance behaviors at the subnational level.The core of the paper revolves around developing a theory that regions reliant on external rents exhibit distinct governance structures. The theory posits that when a region’s economy is heavily supported by income from outside sources, such as natural resource extraction or federal government transfers, it reduces the region's need to tax its citizens. This lack of taxation diminishes the social contract between the state and its citizens, leading to less democratic engagement and greater potential for autocratic governance. The theory draws parallels with national-level rentier states, where governments are similarly less accountable due to their financial independence from the populace. The methodology section outlines the research approach used to investigate the theory. The paper employs a mixed-methods approach, combining both qualitative and quantitative analyses to explore the relationship between external rents and governance characteristics at the subnational level. The study uses statistical models to analyze data from various subnational regions, looking for correlations between the level of reliance on external rents and indicators of governance, such as democratic engagement, accountability, and transparency. In addition to the quantitative analysis, the paper includes case studies of specific subnational regions that are heavily dependent on external rents. These case studies provide a detailed examination of how economic dependency shapes political institutions and governance behaviors in practice. They explore real-world examples of regions that exhibit less democratic engagement and more centralized, non-transparent governance due to their economic reliance on external income. The mixed-methods approach allows the paper to present a comprehensive view of the issue, combining broad statistical trends with in-depth qualitative insights.The findings of the study support the proposed theory, demonstrating a clear link between external rent reliance and specific governance traits at the subnational level. Regions that depend heavily on external rents tend to exhibit less accountability, reduced democratic engagement, and a greater tendency towards centralized, autocratic governance. The analysis reveals that these regions often lack robust institutional frameworks for ensuring transparency and citizen participation, which are more commonly found in regions with diverse economic bases, conclusively emphasizing the importance of understanding economic foundations in shaping political institutions and behavior at the subnational level. It argues that recognizing the impact of external rents on governance can help policymakers design better frameworks for accountability and democratic practices in rentier regions. The paper also suggests that future research should continue to explore these dynamics, particularly in regions where economic dependencies are likely to influence political outcomes significantly.  \par
 \citeA{Skoufias2011} investigate the impact of electoral reforms on local government performance in Indonesia, with a particular focus on the implementation of direct elections for district heads. The study's primary objectives are to assess how these reforms have influenced district-level expenditures and income, to evaluate whether spending patterns align with district needs, and to explore the variations in impact across different regions and leadership situations. By doing so, Bibek aims to understand how electoral accountability shapes fiscal behavior and whether direct elections lead to more responsible and responsive governance at the local level.To address these questions, the study employs a quasi-experimental methodology, comparing districts that transitioned to direct elections with those where elections were postponed. This approach allows for a natural comparison, offering insights into the causal effects of electoral reform. The data are drawn from several sources, including the Village Potential Series, the National Socio-Economic Survey, and district-level budgetary records. These sources provide a comprehensive view of how expenditures, revenues, and budget surpluses evolved before and after the introduction of direct elections.Key variables considered in the analysis include the initial levels of public goods and services, service coverage rates, the economic circumstances of districts, and various political factors that could influence local governance outcomes. By controlling for these variables, the study isolates the specific effects of electoral reforms on fiscal performance.\par
Their results show that the introduction of direct elections led to significant increases in both spending and income at the district level. One of the key findings is that revenue generation from internal sources—such as local taxes and fees—saw substantial growth, suggesting that electoral accountability encouraged district leaders to put greater effort into generating revenue. As a result, districts experienced larger budget surpluses, with revenue growth outpacing the increase in spending. This finding supports the idea that direct elections can promote greater fiscal responsibility, as elected officials are incentivized to manage local budgets more effectively.However, the study also uncovers important regional differences in the effects of electoral reform. The positive impact of direct elections on spending was more pronounced in districts outside of Java and Bali, indicating that these reforms had a stronger influence on fiscal performance in less developed regions. Additionally, the election of new leaders was associated with greater fiscal improvements, while the re-election of incumbents had a less significant impact. This suggests that the introduction of direct elections may have helped bring fresh perspectives and accountability to local governance, particularly in areas where leadership turnover occurred.Despite the overall improvements in fiscal performance, the study finds only a modest alignment between expenditures and district needs. This indicates that while electoral reforms have led to increased spending, the additional funds were not necessarily directed toward addressing specific local challenges. In fact, the reforms may have exacerbated existing imbalances, with some regions benefiting more than others, rather than ensuring that spending was targeted to the areas of greatest need. One notable example is the health sector, which did not see significant increases in expenditures despite the overall growth in spending. This raises concerns about whether certain sectors are being neglected under the new electoral system and suggests the need for further investigation into the allocation of resources.In summary, this study concludes that while electoral reforms in Indonesia have improved fiscal performance and helped increase budget surpluses, they have not fully addressed the issue of regional imbalances or led to significant improvements in service delivery, particularly in the health sector. The study highlights the need for further research into the long-term effects of electoral changes, their impact on specific sectors, and the optimal timing of such reforms to ensure that local governments are both accountable and responsive to the needs of their constituents. These findings underscore the complexity of implementing electoral reforms and the importance of carefully designing policies to maximize their potential benefits. \par
\citeA{Ferraz2011} investigate the link between reelection incentives and corruption among municipal politicians of Brazil. The major goal is to determine if the prospect of reelection successfully punishes incumbent politicians and minimizes their involvement in corrupt activities. To address this, the research employs a unique dataset of corruption practices revealed by local politicians and takes use of the variance in electoral incentives caused by term limitations. It specifically compares first-term mayors, who are eligible for reelection, to second-term mayors, who are subject to a term restriction and hence termed "lame ducks." The data show that first-term mayors are much less corrupt than their second-term counterparts, with 27\% less rent-seeking activity. This difference is consistent across model configurations and compensates for possible confounding variables such as unobserved variations in municipal features, political aptitude, or experience. The paper interprets these findings using a political agency model, implying that first-term mayors are prevented from corrupt acts owing to fear of electoral repercussions. Methodologically, the article employs a variety of factors to examine the consequences of electoral accountability. Key determinants include the existence of local media (radio stations and newspapers) and local prosecutors, who act as proxy for information availability and the perceived cost of corruption. The research also includes indicators of political competitiveness, such as the proportion of local council members from the mayor's party and a political Herfindahl index, to investigate how election dynamics affect corruption levels. Regression studies are performed to compare the effects of these factors on corruption between first- and second-term mayors. To overcome such biases, the study investigates whether first-term mayors are more likely to bribe auditors, or if auditors are swayed by election timing or party connections. The findings show no significant changes in impacts based on audit timing, auditor affiliations, or the value of examined projects, implying that the observed variances in corruption are not due to auditor prejudice. Furthermore, the research investigates the concept that second-term mayors are less motivated to conceal corruption due to their lame-duck position, but concludes that this explanation does not adequately explain for the observed results. This study concludes that electoral accountability, as supported by reelection incentives, has an important role in reducing corruption among municipal officials. The findings lend credence to the idea that measures assuring political accountability may considerably restrain rent-seeking behaviors, especially in environments where corruption is endemic and political elites dominate. However, the report emphasizes the need for more research to establish the best design of term limits and to assess their broader influence on governance, policymaking, and overall voter welfare. The study emphasizes the significance of boosting openness and information availability, as well as political rivalry, in order to increase the effectiveness of electoral accountability in eliminating and reducing corruption. \par
\citeA{Jones2012} examine into the intricate connection between public expenditure and election results, with a focus on federal systems of Argentina. Voters often punish governments that run deficits or raise expenditure, as previous literatures have demonstrated. It has been observed that voters in the United States and other similar nations generally react unfavorably to budgetary growth. This is a commonly held belief in the literature, especially in industrialized countries where voters are thought to place a higher priority on budgetary discipline. This prevalent knowledge, particularly in emerging nations with federal systems, does not always hold true as voters may actually favor fiscal growth at the subnational level. According to previous research, voters in the United States, Israel, and Colombia, for instance, typically penalize municipal governments for having lax budgetary policies. However, in Brazil, Russia, and Argentina, more regional or local government spending is frequently linked to better election results for incumbents, indicating a distinct dynamic at work.The study makes the case that these disparate outcomes are due to the way fiscal federalism is set up in these nations. Voters frequently believe that more public expenditure is funded by the federal government rather than by their own taxes under federal systems like Argentina's. The people are more inclined to reward politicians who participate in fiscal expansion as a result of this impression, which lowers their perception of the actual cost of such expenditure. This conduct resembles the dynamics seen in American pork-barrel politics, in which local legislators get compensation for obtaining federal funding for their districts without raising local taxes in the process. The authors offer a simplified model that compares various budgetary arrangements and how voter behavior is affected in order to explain these events. Voters are more inclined to penalize excessive expenditure in systems with strict budgetary limitations, like the US, since they face the actual expense of the increases. On the other hand, people could be more likely to back politicians who raise expenditure under systems with loose budget restrictions, like Argentina's, where the cost is shared by the entire country.These findings have important policy and governance ramifications. This study, using the result of six gubernatorial elections from 1987-2007 in 24 provinces of Argentina, contends that voter behavior and political incentives to support fiscal expansion can be significantly impacted by the structure of fiscal federalism. This knowledge is essential for developing policies that promote financial restraint while upholding political accountability. Thus, by highlighting the significance of fiscal federalism in influencing voter incentives and election outcomes—particularly in developing nations with decentralized governance structures—the article contributes to the growing body of research on the political economy of fiscal policy.\par
 \citeA{Gemmell2013} explore the impact of fiscal decentralization on economic growth, with a particular focus on how the devolution of spending and revenue responsibilities influences this relationship. The study revisits Oates' (1972) premise, which argues that economic efficiency improves when there is a close alignment between revenue discretion and expenditure obligations at the subnational level. This study aims to provide a more nuanced understanding of this relationship by analyzing how different aspects of fiscal decentralization interact to affect growth.To conduct this analysis, this paper employs panel data regressions and the Pooled Mean Group estimator, enabling the study to account for both long-term and short-term dynamics across countries. By incorporating instrumental variables, such as lagged values of fiscal decentralization and investment, the study addresses potential issues of endogeneity—where variables might be mutually influencing each other. Extensive diagnostic tests confirm the robustness of the results, adding credibility to the conclusions.The study is based on panel data from 23 OECD countries over the period 1972–2005, offering a broad view of how fiscal decentralization has influenced growth in developed economies. It explores various dimensions of decentralization, including government revenue composition and the distinction between direct expenditure controlled by local entities versus self-financed revenue streams. The findings indicate that while expenditure decentralization—particularly when local entities have control over spending without equivalent control over revenues—tends to have a negative influence on economic growth, revenue decentralization has the opposite effect, promoting growth.Specifically, the study supports Oates' hypothesis by demonstrating that when revenue discretion is closely aligned with expenditure obligations at the local level, it leads to more efficient and effective governance, which in turn fosters economic growth. The results are consistent across different datasets and methodologies, reinforcing the conclusion that both spending and revenue decentralization should be considered together to fully understand their effects on growth.One of the key policy implications of the study is that to maximize economic growth, governments should focus on reducing expenditure decentralization—particularly in cases where local entities have significant spending responsibilities without sufficient revenue control—while simultaneously increasing the portion of local expenditures that are self-financed through local revenues. This alignment ensures that local governments are held accountable for their spending and are motivated to use resources efficiently, ultimately contributing to stronger economic performance. The findings highlight the importance of fiscal balance at the subnational level and suggest that a well-designed fiscal framework that ties revenue collection to expenditure obligations can be a powerful driver of economic growth.\par
 \citeA{Gelineau2013} investigates how electoral representatives are accountable to their electorate in developing countries. Not satisfied with the extant literatures which focus heavily in the industrialized nations, this study examines the accountability aspects and electoral incentives in the developing nations. Using a large-n approach with the survey data, this analysis develops a series of incumbent support model to measure the impact of economic assessment in developing countries. This study separates the motives of electorate, and focuses on the economic vote, the voters who actually vote primarily based on the economic aspects. This study uses the public opinion data from various developing nations, estimates incumbent support from it, and estimate average effect using Monte Carlo simulation.  The variables investigated include `economy', which captures individuals' assessments of past and future economic performance; `ses' (socioeconomic status), which includes variables like age, education, and gender; `ideology', which reflects political attachment or ideological stance; and `controls', which include region-specific factors such as corruption, crime, and support for democracy. The dependent variables utilized are `vote' (voting intentions) and `approv' (approval of the president or administration). The findings show that economic assessments regularly influence incumbent support, with negative evaluations leading to disapproval and favorable evaluations to support. This impact is typically significant across both the `vote' and `approv' models, although the `approv' model has a greater rate of significance, most likely due to the difficulties in estimating vote intentions outside of electoral seasonsThis study finds that economic aspects are important factor in holding the elected officials accountable to people, as people heavily base their voting decisions on it. Thus, this study, encompassing 51 countries of the developing world over the 1995-2009 period, concludes that while voters do strongly hold the incumbents accountable for economic aspects, they also sometimes vote on prospective assessments rather than strictly retrospective economic assessments.\par
 \citeA{Porcelli2014} evaluates the efficiency of local governments of Italy and how it relates to the electoral accountability. Previous literatures posit that electoral systems can significantly influence the performance of government by incentivizing politicians to act in the best interest of voters. These models suggest that when voters can directly hold their elected officials accountable, particularly in majoritarian electoral systems, there is a greater likelihood of reducing rent-seeking behaviors and enhancing public service delivery.However, these studies have frequently encountered difficulties in accounting for the complexities and variations in institutional settings, leading to inconclusive or context-specific findings. The need for more robust empirical approaches that can address these challenges is a recurring theme in this study of Italian local governments. Comprehensively examining the reforms of Italian regional governments in health care sector, this study evaluates the impact of the electoral reform in a quasi-experimental setting.Thus, this study builds on this methodological tradition by integrating DiD with stochastic frontier analysis (SFA) and data envelopment analysis (DEA). This combination allows for a more nuanced examination of the efficiency effects of electoral and fiscal reforms, specifically within the context of the Italian health care sector. The study concludes by agreeing with recent theoretical literatures showing a positive association between local government efficiency and the electoral accountability, which are further influenced by the institutional factors such as electoral rules and fiscal decentralization. \par
  \citeA{Fossati2018} analyzes the early theories of electoral accountability which were largely developed in established democracies, where voters are expected to evaluate incumbents based on their performance in office. These theories suggest that democratic accountability is a key mechanism through which citizens can hold elected officials responsible, thus ensuring good governance. However, the application of these theories to developing democracies has been met with mixed results, often due to varying levels of democratic consolidation, political competition, and the presence of clientelism.In Southeast Asia, and Indonesia in particular, the scholarship has been divided on whether electoral accountability functions effectively. Some scholars argue that clientelism and patronage politics dominate the electoral process in Indonesia, undermining the accountability mechanism. These studies emphasize the persistence of money politics, vote-buying, and the influence of local elites in determining electoral outcomes. As a result, voters are often seen as passive recipients of patronage, rather than active agents in holding their leaders accountable.However, other scholars have highlighted the gradual emergence of performance-based voting in Indonesia, particularly in urban areas with higher levels of education and political awareness. These studies suggest that as democracy deepens, voters begin to assess incumbents more critically, taking into account their policy performance and governance outcomes This shift is seen as part of a broader trend towards democratic consolidation, where electoral accountability becomes more effective over time.This study builds on this body of literature by examining electoral accountability in three Indonesian cities: Medan, Samarinda, and Surabaya. The choice of Medan, Samarinda, and Surabaya as case studies is deliberate, as these cities represent different political and socio-economic environments. Medan is characterized by intense political competition and a history of clientelism, while Samarinda presents a middle ground with moderate political contestation. Surabaya, on the other hand, is noted for its relatively high levels of civic engagement and government performance. It contributes to the ongoing debate by providing empirical evidence that while clientelism remains prevalent, there is also significant evidence of performance-based voting. The core of the study's quantitative analysis involves a survey conducted among voters in these three cities. The survey was designed to capture voters' evaluations of local government performance and their voting behavior in recent elections. Respondents were asked about their satisfaction with local services, perceptions of corruption, and their likelihood of supporting the incumbent mayor or governor. To complement the quantitative findings, the study also incorporates qualitative interviews with key informants, including local politicians, campaign managers, and political analysts. These interviews provide context to the survey data by exploring the dynamics of local politics, the role of clientelism, and the strategies used by incumbents to secure votes. The study’s findings suggest that Indonesian voters are increasingly using their electoral power to hold local politicians accountable, indicating a more nuanced understanding of how democracy is functioning at the local level in Indonesia. This research thus aligns with the emerging view that electoral accountability is possible in developing democracies, but its effectiveness varies depending on local contexts and the level of democratic maturity.\par
  \citeA{Veiga2019}  investigate how accountability, driven by electoral incentives, shapes fiscal policies in Portugal’s local governments, particularly through the imposition of mayoral term limits. The study aims to address conflicting findings in the literature regarding how accountability is impacted when term limits are introduced. By focusing on 308 municipalities over five election cycles (1998–2013) and using a difference-in-differences approach, the study examines how term limits affect fiscal outcomes such as revenues and expenditures.The findings reveal that when mayors are term-limited and unable to run for reelection, accountability diminishes. These mayors tend to reduce efforts to collect revenues or sustain spending, leading to a decline in both. Without the electoral pressure to appeal to voters, their incentives to engage in proactive fiscal management weaken, resulting in a notable decrease in government performance.Conversely, mayors who are eligible for reelection are found to be more accountable, as they are motivated by electoral incentives to secure voter approval. These mayors focus on new investments and actively pursue conditional grants from the central government, which can directly impact their chances of reelection.The study concludes that elections play a key role in holding officials accountable for their fiscal policies. However, the introduction of term limits can undermine this accountability, as the lack of reelection prospects reduces the motivation for mayors to maintain fiscal discipline. In this way, while term limits are intended to prevent political entrenchment, they can inadvertently weaken the fiscal responsibility of local leaders.\par
 \citeA{Marin2021} investigate the link between electoral participation, political competitiveness, and fiscal success in Colombian municipalities between 2000 and 2015. The research uses an unbalanced panel data fixed effects model to compensate for unobserved time-invariant local institutional structures, ensuring that the analysis takes into account each municipality's distinctive characteristics.Municipal fiscal performance, as measured by the composite Fiscal Performance Index (FPI), is the dependent variable, with political competitiveness and election participation serving as the key independent factors. The FPI is calculated using Principal Components Analysis (PCA) and takes into account six essential areas of local public finances: operational costs, debt service, transfer revenue, internal resource creation, investment spending, and saves capacity. Political competitiveness is evaluated by the Herfindahl-Hirschman Index (HHI) and the mayor's party's percentage of council seats, whilst voter turnout rates show electoral involvement. Control factors such as political affiliation with higher levels of government, education quality, violence occurrence, income, and population density are included to offer a more complete study. The findings show that more political competition, as measured by a lower HHI, and increased voter turnout have a favorable impact on local budgetary performance. Specifically, greater distributed political power within the council and active voter engagement result in better budgetary outcomes, which is consistent with political accountability theories that believe higher competition and scrutiny encourage better governance. Furthermore, the examination of FPI subcomponents reveals that political competition improves own resource production, investment, and savings capacity while decreasing operational expenditures and debt servicing costs.The study recognizes a constraint in completely capturing political competitiveness in Colombia due to the fluid nature of party memberships, where politicians regularly switch parties, confounding the examination of formal electoral statistics. Despite this, the study concludes that the decline of traditional political party dominance and the rise of new parties over the last two decades has benefited Colombia's local fiscal performance, emphasizing the importance of political diversity and active citizen participation in improving governance outcomes.\par
\citeA{Mendoza2021} examine the impact of government performance and expenditure patterns on the likelihood of mayors being recalled or reelected in Peruvian municipalities from 2011 to 2014. The study seeks to comprehend how various performance measures and spending patterns influence local political results. To do this, a cross-sectional study was carried out on a sample of 1,632 district municipalities, using ordered logit models to investigate recall probability and probit binary models to estimate reelection prospects. The recall dependent variable is an ordinal variable that indicates if a Mayor is not part of a recall process, is part of the recall process but not revoked, or has been revoked. The reelection model employs a binary dummy variable that indicates whether a Mayor was reelected or not. The research takes into account performance characteristics such as garbage collection frequency and primary school dropout rates, both of which are important in determining the efficacy of municipal services. Several models are used to examine spending variables, including total expenditure per capita, transfer-financed expenditure, and the distinction between capital and current expenditures. The data show that improved waste management and decreased dropout rates have a major impact on both recall and reelection. improved overall expenditure per capita is linked to lower recall hazards and improved reelection prospects. However, the analysis reveals an unexpected finding: more dependence on vertical transfers, rather than improving reelection prospects, is associated with a higher probability of remembering. This shows that voters may see local governments' reliance on transfers as evidence of poor performance. The data also reveals that political alignment with higher levels of government and the number of candidates in elections increase recall probability, while biases such as gender discrimination against female mayors and geographical accessibility issues influence outcomes even more. The study suggests that, while municipal accountability systems in Peru are relatively efficient in holding mayors accountable based on performance criteria, flaws exist, including the possibility of political misuse and prejudice. \\
\vspace{-9mm}
 \subsection{Literature Gap}
 As Nepal is a relatively new federal state, paucity of research have been carried out in the national context examining the role of electorate in shaping the fiscal discipline of their local government. Neither have there been literatures analyzing the similarities and disparities that exist among local level of Nepal. Existing theoretical and empirical literature suggests that electorate influence the behavior of the elected officials in regard to the formulation and execution of fiscal and economic policies. However, such conclusions are not always valid as other literature contends that electoral accountability of officials depends on the institutional and other factors and that the officials might have little to no accountability to the electorate.\par
 Further, federalism, with different level of governments, complicates the accountability process as voters might not have enough information to adjudge the policy formulation and implementation jurisdictions of different level of governments. In Nepali context, local level governments are the closest level of government and its officials are elected by the citizens of that local level. With the clear link in existing literature between fiscal and economic discipline with the economic growth, it is paramount to analyze the extent to which the electorate affects the fiscal and economic decision making of the local level entities of Nepal.