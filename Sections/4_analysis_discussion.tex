\newpage
\section{ANALYSIS AND DISCUSSION}
In this section, descriptive and explanatory analysis will be conducted on the variables of interest. Descriptive analysis details the measure of central tendency as well as the measure of dispersion of the independent, control, and dependent variables. The results of the OLS regression will be presented and the relationship among the variables will be analyzed in the explanatory subsection.
\vspace{-3mm}
\subsection{Descriptive Analysis}
The Mean, Standard deviation, minimum, and maximum value of the variables will be presented in Table 2
These statistics help to understand how the overall data is distributed, which will be further analyzed in the explanatory analysis.
\begin{table}[H]
  \centering
   \begin{tabular}{lcccc}
    \hline
    \multirow{2}{*}{Variables} &\multicolumn{4}{c}{Summary Statistics}\\&{$Mean$} & {$Std. Dev.$}  & {$Min$} & {$Max$} \\
        \hline
    HHI & 0.351 &0.101&  0.201  & 1\\
    VoterTurnout & 74.475 &5.821 &  41.501&100 \\
    HighEdu & 0.147 & 0.059& 0.032& 0.491\\
    IntRevtot &0.035 &0.048 &0 &0.424\\
    RevShrtot & 0.135 &0.039& 0.031  & 0.451\\
    CurExp & 0.512& 0.089& 0.198   & 0.778\\
    Berujupc & 2.139& 4.5721& 0.021  & 115.197\\
    \hline
    \end{tabular}
    \caption{Descriptive Satistics of Variables }  
     \label{Descriptive Statistics of Variables }
\end{table} 
The descriptive statistics gives context to the variables of interest. Firstly, two dummy variables, namely DumRM, which represents whether the local level is a Rural-Municipality or not , and Divcontrol, which indicates if the local area government is controlled by the single party or not, is not indicated in the table above. There are 460 Rural-Municipalities in Nepal, and the rest are either municipalities, sub-metropolitan, or metropolitan city, thus, DumRM has 460 dummy as 1, and rest as 0. Similarly, there are 183 local level governments with divided political control at local level governments based on the election of 2017. \\
As seen from the table, the HHI average value is 0.351, meaning that overall the electoral competition was competitive at the mayoral level of Nepal. The minimum value of HHI, indicating more competitive election was 0.201, which in a five party framework denotes extreme competition. The maximum value of 1, which indicates no competition, was due to the fact that in three of the local jurisdiction has unopposed mayors, indicating no competition. For voter turnout variable, the average is 74.475, with the standard deviation of 5.821. The maximum value of 100 is generated because of the unopposed mayoral candidates in the election. For higher education variable, it shows how uneven the educational attainment is among the local levels, as in some local levels, only about 3 \% of people have intermediate degree of more, whereas the maximum value is nearly of majority having such degree.\\
The dependent variables also show interesting dynamics. These variables were generated by averaging their performance of fiscal years of 2076/77,2077/78, and 2078/79 B.S., so that cross-sectional regression could be performed, while losing the time dimension. The averaged IntRevtot, which shows the revenue generated internally to the total revenue has the value of 0.035, meaning on average only about three and half percent of total revenue are generated internally, while in some local-levels, more than 42\% of the total revenue was generated internally. Similar is the case of revenue sharing to total revenue variable and current expenditure variable. In CurExp, the local levels spend more than half of their total expenditure as current expenditure, which is concerning. Irregularity variable, Berujupc, shows how some local-levels have done well to control such expenditures while others have more to do, and how spread out the distribution is.
\subsection{Electoral influences on fiscal discipline in Nepal}
\subsubsection{Correlation Analysis}
In this section, the correlation among the regressor variables are analyzed to ensure that there is no multicollinearity among the regressor. Since our regressors are HHI, VoterTurnout, Highedu, Divcontrol, and DumRM, the correlation Analysis will look at the correlation among them, as shown is the Table 3.\\
\begin{table}[ht]
\centering
\begin{tabular}{cccccc}

    & HHI & VoteTurnout & Highedu & DumRM & Divcontrol \\ 
HHI & 1.000 &  &  &  &  \\ 
VoteTurnout & -0.110 & 1.000 & &  &  \\ 
Highedu & 0.087 & 0.084 & 1.000& &  \\ 
DumRM & 0.092 & -0.097 & -0.3726 & 1.000 & \\ 
Divcontrol & 0.095 & 0.047 & 0.042 & -0.049 & 1.000 \\ 
\end{tabular}
\caption{Correlation Matrix}
\label{Correlation Matrix}
\end{table}
From Table, there seems to be no high correlation among variables for the problem of multi-collinearity to arise. At the higher end of spectrum of correlation is the correlation between the dummy for Rural-Municipalities and higher education variable of -0.372, indicating that higher education value is moderately influenced by whether the local level jurisdiction is a municipality or not. \\
The following table checks for the multicollinearity using the VIF. 
\begin{table}[ht]
\centering
\begin{tabular}{cccc}

     Variables & VIF & 1/VIF \\ 
HHI & 1.07 &0.938    \\ 
VoteTurnout & 1.03& 0.971  \\ 
Highedu & 1.19 & 0.838  \\ 
DumRM &1.20 & 0.830 \\ 
Divcontrol &1.02 & 0.984 \\ 
Year.2077 & 1.33& 0.750\\
Year.2078 & 1.22 & 0.7500\\
Mean VIF & 1.17 & \\
\end{tabular}
\caption{VIF- Multicolinearity Test}
\label{VIF- Multicolinearity Test}
\end{table}
As the above table shows, this study would not have concern with implementing pooled regression model as the variables are not correlated to each other, allowing for a more reliable coefficient estimates.
\subsubsection{Results from Cross-Sectional Regression Analysis}
Using the equation from the model specification section, this study carried out the OLS linear equation to explore the relationship between independent and dependent variables. The dependent variable(Y) are Intrevtot, Revshrtot, Curexp, and Berujupc, while independent variables(X) are HHI, VoteTurnout, Highedu, DumRM and Divcontrol. The result of the regressions are given in the table below.\\
\begin{table}[ht]
\centering
\begin{tabular}{|ccccc|}
\hline
\textbf{Variables} & \textbf{Intrevtot} & \textbf{Revshrtot} & \textbf{Curexp} & \textbf{Berujupc} \\
\hline
HHI & \makecell{-0.024* \\ (0.014)} & \makecell{-0.044*** \\ (0.013)} & \makecell{0.074 **\\ (0.032)} & \makecell{10.099*** \\ (1.649)} \\

VoteTurnout& \makecell{0.03 \\ (0.000)} & \makecell{0.001* \\ (0.002)} & \makecell{0.02***\\ (0.000)} & \makecell{0.092*** \\ (0.028)} \\

Highedu & \makecell{0.395*** \\ (0.026)} & \makecell{0.192*** \\ (0.025)} & \makecell{-0.096* \\ (0.058)} & \makecell{-5.026* \\ (3.012)} \\
DumRM & \makecell{-0.017*** \\ (0.003)} & \makecell{0.002\\ (0.003)} & \makecell{0.016**\\ (0.007)} & \makecell{0.483 \\ (0.362)} \\
Dividedcontrol & \makecell{0.002 \\ (0.003)} & \makecell{-0.002\\ (0.003)} & \makecell{0.013*\\ (0.007)} & \makecell{0.698*\\ (0.378)} \\
Constant & \makecell{-0.029 \\ (0.020)} & \makecell{0.035* \\ (0.019)} & \makecell{0.671** \\ (0.044)} & \makecell{-7.986*** \\ (2.263)} \\\hline

Observations & 753 & 753 & 753 & 753 \\
\hline
R-squared &0.336 & 0.121 & 0.062 & 0.071 \\
\hline
\footnotesize Standard errors& \footnotesize in parentheses&, \footnotesize***p$<$0.01, &\footnotesize **p$<$0.05, *p$<$0.1\\
\end{tabular}
\caption{Cross-Sectional Regression results with coefficients and standard errors}
\label{tab:coefficients_with_se}
\end{table}
As the above table shows the regression output and thus, the relationship among variables. HHI, the index of electoral competition is significant across all the dependent variables, and the direction is as the study expected in the earlier section. As electoral competition decreases, accountability weakens and the internal revenue of the local government suffers at the significant level. Same is the case with revenue sharing mechanism, and the result is more significant here. For current expenditure and irregularities variables, the sign is as expected and the relationship is established at the significant level. Thus, this study contends that electoral competition does indeed strengthens the electoral accountability and the indicators of the fiscal health.\\
Voting turnout is also the measure of electoral significance and it affects the dependent variable in the expected way. As turnout increases, the internal revenue do increase, but not at the significant level, whereas revenue share does increase in small magnitude at a weakly significant level. Voting turnout do significantly increases the current expenditure and beruju, and the direction of influence is different than what this study initially posited. It could be the result of the local level governments to increase current expenditure to give favors to the people close to them after the elections through current expenditures or through some other unobserved mechanism.\\
Higher education influence in all fiscal dependent variables is significant. Higher education significantly influences the internal revenue and revenue sharing magnitude of the local governments and also influences significantly, though weakly, the expenditure variables.\\
The effect of dummy variable of Municipalities significantly affects internal revenue collection and current expenditure. It can be observed that the local jurisdiction being a rural-municipality negatively affects the internal revenue collection at the local level significantly. Similarly, current expenditure also increases at the significant level, along with beruju at non-significant level.
The effect of having divided control of the local government structure is not pronounced for the dependent variables, although it does have a weakly significant and positive effect on the current expenditure and beruju.\\
The R-Squared measure of different equations are provided at the end of the table. While the income dependent variables are moderately explained by the independent variables, the expenditure dependent variables are weakly explained by the dependent variables. It should be noted that in economics, like in other social science disciplines, the behaviors of people are complex and the explanatory power of model of human behaviors are usually at the lower end of the spectrum.\\
\subsection{Objective 2 related: Results from Pooled-Regression Analysis}
\begin{table}[ht]
\centering
\begin{tabular}{|ccccc|}
\hline
\textbf{Variables} & \textbf{Intrevtot} & \textbf{Revshrtot} & \textbf{Curexp} & \textbf{Berujupc} \\
\hline
HHI & \makecell{-0.0243** \\ (0.009)} & \makecell{-0.044*** \\ (0.009)} & \makecell{0.074 **\\ (0.029)} & \makecell{10.099*** \\ (6.044)} \\

VoteTurnout& \makecell{0.0003** \\ (0.000)} & \makecell{0.001*** \\ (0.000)} & \makecell{0.02***\\ (0.000)} & \makecell{0.092*** \\ (0.028)} \\

Highedu & \makecell{0.395*** \\ (0.045)} & \makecell{0.192*** \\ (0.439)} & \makecell{-0.096* \\ (0.049)} & \makecell{-5.026* \\ (4.225)} \\
DumRM & \makecell{-0.017*** \\ (0.002)} & \makecell{0.002\\ (0.002)} & \makecell{0.016**\\ (0.005)} & \makecell{0.483 ***\\ (0.179)} \\
Dividedcontrol & \makecell{0.003 \\ (0.002)} & \makecell{-0.002\\ (0.002)} & \makecell{0.013*\\ (0.007)} & \makecell{0.699*\\ (0.363)} \\
2077.Year& \makecell{0.004 \\ (0.003)} & \makecell{0.013***\\ (0.002)} & \makecell{0.043***\\ (0.007)} & \makecell{0.440\\ (0.366)} \\
2078.Year & \makecell{0.0017 \\ (0.002)} & \makecell{0.029***\\ (0.002)} & \makecell{0.050***\\ (0.007)} & \makecell{0.134\\ (0.366)} \\
Constant & \makecell{-0.031 \\ (0.128)} & \makecell{0.021 \\ (0.014)} & \makecell{0.640** \\ (0.037)} & \makecell{3.06 \\ (0.049)} \\\hline

Observations & 2259 & 2259 & 2259 & 2259 \\
\hline
R-squared &0.2396 & 0.125 & 0.056 & 0.049 \\
\hline
\footnotesize Standard errors& \footnotesize in parentheses&, \footnotesize***p$<$0.01, &\footnotesize **p$<$0.05, *p$<$0.1\\
\end{tabular}
\caption{Pooled Regression results with coefficients and robust standard errors}
\label{Pooled Regression results with coefficients and robust standard errors}
\end{table}



\subsection{Discussion}

