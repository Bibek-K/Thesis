\newpage
\section{ANALYSIS AND DISCUSSION}
In this section, descriptive and explanatory analysis will be conducted on the variables of interest. Descriptive analysis details the measure of central tendency as well as the measure of dispersion of the independent, control, and dependent variables. The results of the p regression will be presented and the relationship among the variables will be analyzed in the explanatory subsection.
\vspace{-3mm}
\subsection{Descriptive Analysis}
The Mean, Standard deviation, minimum, and maximum value of the variables will be presented in Table 2
These statistics help to understand how the overall data is distributed, which will be further analyzed in the explanatory analysis.
\begin{table}[H]
  \centering
   \begin{tabular}{lcccc}
    \hline
    \multirow{2}{*}{Variables} &\multicolumn{4}{c}{Summary Statistics}\\&{$Mean$} & {$Std. Dev.$}  & {$Min$} & {$Max$} \\
        \hline
    HHI & 35.058 &10.084&  20.091  & 100\\
    VoterTurnout & 74.475 &5.818 &  41.5&100 \\
    HighEdu & 14.691 & 5.839& 3.161& 49.091\\
    Employ & 22.393 & 13.103 & 5.099 & 67.257\\
    IntRevtot &3.504 &5.677 &0 &71.135\\
    RevShrtot & 13.547 &5.216& 0  & 56.451\\
    CurExp & 51.209& 13.162& 0  & 90.627\\
    Beruju &9.441& 12.442& 0  & 4.172\\
    TotScore & 58.930 & 15.601 & 5.25 & 97.5\\
    GovtProv & 80.030 & 14.871 & 0 & 100\\
    BudgPlan & 61.157 & 18.046 & 0 & 100\\
    ServDeliv & 63.995 & 18.919 & 0 & 100\\
     PhyInfra & 43.749 & 21.219 & 0 & 100\\
    \hline
    \end{tabular}
    \caption{Descriptive Satistics of Variables }  
     \label{Descriptive Statistics of Variables }
\end{table} 
The descriptive statistics gives context to the variables of interest. Firstly, two dummy variables, namely DumRM, which represents whether the local level is a Rural-Municipality or not , and Divcontrol, which indicates if the local area government is controlled by the single party or not, is not indicated in the table above. There are 460 Rural-Municipalities in Nepal, and the rest are either municipalities, sub-metropolitan, or metropolitan city, thus, DumRM has 460 dummy as 1, and rest as 0. Similarly, there are 183 local level governments with divided political control at local level governments based on the election of 2017. \par
As seen from the table, the HHI average value is 35.058, meaning that overall the electoral competition was fairly competitive at the mayoral level of Nepal. The minimum value of HHI, indicating more competitive election was 0.201, which in a five party framework denotes extreme competition. The maximum value of 1, which indicates no competition, was due to the fact that in three of the local jurisdiction has unopposed mayors, indicating no competition. For voter turnout variable, the average is 74.475, with the standard deviation of 5.818. The maximum value of 100 is generated because of the unopposed mayoral candidates in the election. For higher education variable, it shows how uneven the educational attainment is among the local levels, as in some local levels, only about 3 \% of people have intermediate degree of more, whereas the maximum value is nearly of majority having such degree.\par
The variables of fiscal discipline also show interesting dynamics. These variables were generated though the performance of the local levels in fiscal years of 2076/77,2077/78, and 2078/79 B.S., so that pooled regression could be performed, while losing the panel dimension for the first regression. The variable IntRevtot, which shows the revenue generated internally to the total revenue has the value of 3.504, meaning on average only about three and half percent of total revenue are generated internally, while in some local-levels for some years, more than 71\% of the total revenue was generated internally. Similar is the case of revenue sharing to total revenue variable and current expenditure variable. In CurExp, the local levels spend more than half of their total expenditure as current expenditure, which is concerning. Irregularity variable, Beruju, shows how some local-levels have done well to control such expenditures while others have more to do, and how spread out the distribution is.
Similarly, the variables of the government performance scores on different aspects of local governments, also shows interesting dynamics, The total scores of the local governments in the three year periods ranges from 5.25 to 97.5, with a mean score of 58.930. Other components of the total scores are normalized and shows how they differ across the local governments in Nepal.
\subsection{Electoral influences on governance performance of local levels}
\subsubsection{Correlation Analysis}
In this section, the correlation among the regressor variables, for the pooled regressions, are analyzed to ensure that there is no multicollinearity among the regressor. Since our regressors are HHI, VoterTurnout, Highedu, Divcontrol, Employ, and DumRM, for the first regression, the correlation Analysis will look at the correlation among them, as shown is the Table 3.\\
\begin{table}[ht]
\centering
\begin{tabular}{ccccccc}

    & HHI & VoteTurnout & Highedu & DumRM & Divcontrol & Employ\\ 
    \hline
HHI & 1.000 &  &  &  & &  \\ 
VoteTurnout & -0.1195 & 1.000 & &  & &  \\ 
Highedu & 0.087 & 0.084 & 1.000& & & \\ 
DumRM & 0.1291 & -0.097 & -0.372 & 1.000 & & \\ 
Divcontrol & 0.095 & 0.047 & 0.042 & -0.049 & 1.000 & \\ 
Employ & -0.380 & 0.328 & 0.434 & -0.397 & 0.001 & 1.000 \\
\hline
\end{tabular}
\caption{Correlation Matrix}
\label{Correlation Matrix}
\end{table}
From Table, there seems to be no high correlation among variables for the problem of multi-collinearity to arise. At the higher end of spectrum of correlation is the correlation between the employment  and higher education variable of 0.434, indicating that higher education value is moderately influenced by the employment variable value. As the above correlation values shows the mild correlation of Employ with other variables, this study also checks for VIF.   \par
The following table checks for the multicollinearity using the VIF. 
\begin{table}[ht]
\centering
\begin{tabular}{cccc}

     Variables & VIF & 1/VIF \\ 
     \hline
HHI & 1.28 &0.781    \\ 
VoteTurnout & 1.15& 0.991  \\ 
Employ & 1.88 & 0.531\\
Highedu & 1.52 & 0.658  \\ 
DumRM &1.28 & 0.781 \\ 
Divcontrol &1.01 & 0.990 \\ 
Year.2077 & 1.47& 0.681 \\
Year.2078 & 1.47 & 0.681\\
\hline
Mean VIF & 1.38 & \\
\end{tabular}
\caption{VIF- Multicolinearity Test}
\label{VIF- Multicolinearity Test}
\end{table}
As tables 3 and 4 show, this study would not have concern with implementing pooled regression model as the variables are not correlated to each other, allowing for a more reliable coefficient estimates.
\
\subsubsection{Results from Pooled Regression Analysis}
Using  equation(1) from the model specification subsection of Research Methodology section, this study carried out pooled regression to explore the relationship between independent and dependent variables. The dependent variable(Y) are the scores of the local government performance, including TotScore, GovtProv, ServDeliv, BudgPlan, and PhyInfra,  while independent variables(X) are HHI, VoteTurnout, Highedu, DumRM, Employ and Divcontrol. The result of the regressions are given in the table(5).\\
\begin{table}[ht]
\centering
\begin{tabular}{|cccccc|}
\hline
\textbf{Variables} & \textbf{TotScore} & \textbf{GovtProv} & \textbf{ServDeliv} & \textbf{BudgPlan} & \textbf{PhyInfra} \\
\hline
HHI & \makecell{0.147*** \\ (0.048)} & \makecell{0.175*** \\ (0.045)} & \makecell{0.158***\\ (0.060)} & \makecell{0.227*** \\ (0.053)} & \makecell{0.188*** \\ (0.066)} \\

VoteTurnout& \makecell{-0.226*** \\ (0.076)} & \makecell{-0.114* \\ (0.067)} & \makecell{-0.312***\\ (0.087)} & \makecell{-0.275*** \\ (0.085)}& \makecell{-0.235** \\ (0.111)} \\

Highedu & \makecell{0.617*** \\ (0.094)} & \makecell{0.438*** \\ (0.086)} & \makecell{0.574*** \\ (0.109)} & \makecell{0.434*** \\ (0.105)} & \makecell{0.655*** \\ (0.130)}\\
DumRM & \makecell{-3.293*** \\ (0.999)} & \makecell{-2.210**\\ (0.931)} & \makecell{-2.517***\\ (1.121)} & \makecell{-0.969 \\ (0.1.162)} & \makecell{-7.724*** \\ (1.412)}\\
Employ & \makecell{-0.095* \\ (0.048)} & \makecell{-0.095*\\ (0.046)} & \makecell{-0.053\\ (0.056)} & \makecell{-0.103* \\ (0.056)} & \makecell{-0.060 \\ (0.060)}\\
Dividedcontrol & \makecell{-0.796\\ (0.972)} & \makecell{-0.692\\ (0.917)} & \makecell{-0.060\\ (1.160)} & \makecell{-0.150\\ (1.121)} & \makecell{-1.524 \\ (1.345)}\\
2077.Year& \makecell{7.41***\\ (0.59)} & \makecell{6.40***\\ (0.69)} & \makecell{10.53***\\ (0.80)} & \makecell{7.16***\\ (0.8)} &\makecell{6.59*** \\ (0.82)} \\
2078.Year & \makecell{12.72*** \\ (0.65)} & \makecell{9.86***\\ (0.74)} & \makecell{17.32***\\ (0.85)} & \makecell{11.99***\\ (0.82)} &\makecell{11.16*** \\ (0.92)} \\
Constant & \makecell{58.35*** \\ (5.863)} & \makecell{73.5*** \\ (5.269)} & \makecell{65.63*** \\ (6.715)} & \makecell{71.56*** \\ (6.63)} & \makecell{44.78*** \\ (8.66)}\\
 \hline
Observations & 1998 & 1998 & 1998 & 1998 & 1998 \\
\hline
R-squared &0.1838 & 0.1219 & 0.1816 & 0.1172 & 0.1288  \\
\hline
\footnotesize Clustered Standard & \footnotesize errors& \footnotesize in parentheses&, \footnotesize***p$<$0.01, &\footnotesize **p$<$0.05, *p$<$0.1\\
\end{tabular}
\caption{Pooled Regression results with coefficients and standard errors}
\label{tab:coefficients_with_se}
\end{table}
Table 5 shows the regression output and thus, the relationship among variables. HHI, the index of electoral competition, is highly significant across all the dependent variables, and the direction is as the study expected in the earlier sections. Our regression table agrees with the realm of literature which concede that high electoral competition produces winners who are not guaranteed to win the next election, thus, they aim to maximize their own rent seeking and profiteering behavior while in office. Thus, as HHI increases, the electoral competition decreases, and it would cause the increase in the total score and other scores of the governance performance of local levels as less competitive jurisdiction mayors perceive more chances of winning and are motivated to govern in an accountable and responsible fashion.  Voting Turnout also plays the similar role as it shows significant and negative association with the scores of governance performance. As Voting turnout increases, more people are directly involved in the selection of the candidates, the winning candidates might infer that it would be difficult to win the next election, and thus involve themselves with maximizing the rent seeking behavior. Jurisdiction where turnout is low, seems to encourage mayors to perform better with the hope of being reelected in the next election. \par
The presence of highly educated people also affects the scores of governance performance in a positive and significant manners. More presence of educated people would better use their `sanction and selection' power, thus, encouraging the mayors to improve performance metrics as well as encouraging the losing candidates to better analyze the decisions of the mayors and to  prepare well for the next election. The dummy variable for the rural and municipal jurisdiction significantly affects the performance scores of the government. As the table shows, being a rural-municipality would lower the total score and the scores across other performance variables of the local governments, more so than other variables of the regression model. \par
There is a significant and positive effect of years in the scores of the governance performance measures in this model. It could be because of bevy of factors but it could chiefly be because of the fact that the data keeping and entry methods were improved in subsequent years, and that the impact of COVID-19 were extreme on the beginning of the time frame of data collection of governmental scores. The R-Squared measure of different equations are provided at the end of the table. The R squared values are relative low, as there could be other variables that are missing from this model.  It should be noted that in economics, like in other social science disciplines, the behaviors of people are complex and the explanatory power of model of human behaviors are usually at the lower end of the spectrum.\\
\subsection{Governance performance influence on fiscal discipline - using panel regression result here:show  , with Hausman test, BP LM tests, stationary test and other tests here.}
This section performs the panel regression to investigate the relationship between the score of the governance performance and the variables of fiscal management and discipline. First, correlation and VIF analysis will be conducted to test for multicollinearity among the regressor, then Breusch-Pagan Lagrange Multiplier test would be carried out for for determining between pooled OLS regression and panel regression, followed by Hausman test to determine whether random effect or fixed effect panel regression should be used. Also, Heteroskedasticity and serial autocorrelation will be tested along with cross-sectional dependance and unit root tests. 
\subsubsection{Correlation Analysis}
In this section, the correlation among the regressor variables are checked to ensure that there is no excessive multicollinearity among them. Since our regressors are Total Score, Employment, DumRM, Highedu, and Divcontrol, the correlation Analysis will look at the correlation among them, as shown is the Table 6.\par
\begin{table}[ht]
\centering
\begin{tabular}{cccccc}

    & Total\_Score & Employ & DumRM & Highedu & Divcontrol \\ 
    \hline
Total\_Score & 1.000 &  &  &  &   \\ 
Employ & 0.017 & 1.000 & &  &  \\ 
DumRM &- 0.141 & -0.407 & 1.000& &  \\ 
Highedu & 0.221 & 0.471 & -0.389 &  1.000&  \\ 
Divcontrol & 0.000 & -0.006 & -0.041 & 0.035 &1.000   \\ 
\hline

\end{tabular}
\caption{Correlation Matrix for Panel variables}
\label{Correlation Matrix-Panel}
\end{table}
From Table 6, there seems to be no high correlation among variables for the problem of multi-collinearity to arise, as the variables are similar to those used in correlation analysis of Table 3. Explanations for the relatively high correlation between selected variables are already given previous sections and thus, this study now checks for VIF. Panel regressions account for variability across both cross-sectional units and time, which complicates the calculation and interpretation of the VIF. Thus, this study calculates VIF based on the pooled regression approach as it provides a good approximation for the multicollinearity among variables while conducting panel regression. \par
\begin{table}[ht]
\centering
\begin{tabular}{cccc}

     Variables & VIF & 1/VIF \\ 
     \hline
Highedu & 1.43 &0.697  \\ 
Employ & 1.41  & 0.707\\
DumRM &1.29 & 0.776\\ 
Total\_Score & 1.07& 0.933  \\ 
Divcontrol &1.00 & 0.996 \\ 
\hline
Mean VIF & 1.24 & \\
\end{tabular}
\caption{VIF- Multicolinearity Test for panel regression}
\label{VIF- Multicolinearity Test for panel regression}
\end{table}
As Tables 6 and 7 show, there is no considerable degree of multi-correlation among the variables to be used as the regressors. Thus, we can safely proceed towards checking if the pooled regression would be sufficient or the panel regression would be needed in the next sub-section.\par
\subsubsection{Breusch-Pagan LM Test}
Breusch-Pagan LM test would check whether the data demands the panel regression  or the pooled regression would suffice. This test checks if the cross-sectional units in the panel has the same intercept or a different intercept. Thus, we run both, pooled and random effect panel regression and determine this test statistic to determine which regression model is better suited for our analysis. This test yields the following result:\par
\begin{table}[H]
  \centering
   \begin{tabular}{c|cc}
     Variables &Chibar2(01) & \(prob>chibar2\) \\
        \hline
    Intrevtot & 332 &0.000\\
    Revshrtot & 203.25 &0.000 \\
    Curexp & 33.71 & 0.000\\
    Beruju & 109.2 & 0.000 \\
    \hline
    \end{tabular}
    \caption{Breusch-Pagan LM Test }  
     \label{Breusch-Pagan LM Test }
\end{table} 
As the result indicate that p-value = 0.000, for all the dependent variables , we conclude that the Pooled OLS model is not efficient estimator for our analysis. Thus, this study would be using panel regression model to explain the relationship between the governance scores of the local levels and the fiscal performance.
\subsubsection{Hausman Test}
Breusch-Pagan LM test suggests that the data is not pool-able and panel regression must be conducted. We administer Hausman test to further determine which type of panel regression, fixed effect or random effect, is more suited for the analysis. Since our independent variables contain both time variant and time invariant characteristics, it is essential that we test which model, random or fixed effect would be more appropriate for the analysis. \par
Hausman test helps decide between fixed and random effect model by testing if the unobserved heterogeneity are correlated with the regressors. In the fixed effect model, it is assumed that individual specific effects are correlated with the independent variables, thus, it drops time-invariant regressors, while random effect model assumes that individual specific effects are uncorrelated with the independent variables, thus, it doesn't drop the time-invariant variables, and include those individual effects as a part of error term. Thus, we first run the fixed effect regression, store its estimates, then run the random effect regression and store its estimates, then perform the Hausman test on the stored estimates to determine which regression model is more appropriate. In this test, lower p-value, usually less than 0.05, indicates that fixed effect model is more appropriate, while higher p-value indicates that random effect regression is better suited. The result of the Hausman test is as follows: \par
\begin{table}[H]
  \centering
   \begin{tabular}{c|cc}
     Variables &Chi2(1) & \(prob>chi\) \\
        \hline
    Intrevtot & 0.2 &0.6581\\
    Revshrtot & 2.84 &0.092\\
    Curexp & 33.71 & 0.000\\
    Beruju & 109.2 & 0.000 \\
    \hline
    \end{tabular}
    \caption{Hausman Test }  
     \label{Hausman Test }
\end{table} 
\subsection{Discussion}

